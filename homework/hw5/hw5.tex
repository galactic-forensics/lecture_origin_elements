\documentclass[letterpaper,12pt,twoside=false,DIV=11]{scrartcl}

%----------------------CONFIG---------------------------
%math packages
\usepackage{amsmath,amssymb,amsthm,units,unitsdef}
\usepackage{wasysym}  % astro symbols

%bibliography style and citation style, bibstyles to use: plainnat, abbrvnat, unsrtnat, named, chicago
%otherwise numerical citationstyle will be used
%\usepackage[authoryear,round]{natbib}

\usepackage{longtable,tabularx,tabulary,multirow,lscape}
\usepackage[font={sl},format=plain,labelfont=bf]{caption}

% colors
\usepackage{color,colortbl}
\usepackage[dvipsnames]{xcolor}
\definecolor{darkblue}{HTML}{00354C}

\usepackage{booktabs}
%\usepackage{showkeys} % shows the labels above the references for

%easier development
\usepackage{ifpdf}

\ifpdf
    \usepackage[pdftex]{graphicx}
    \usepackage[]{pdfpages} %for including full pdf pages
    \usepackage[pdftex,
        bookmarks,
        bookmarksopen=true,
        bookmarksnumbered=true,
        pdfauthor={Reto Trappitsch},
        pdftitle={On the origin of elements in the Milky Way - Homework},
        colorlinks,
        linkcolor=darkblue,
        citecolor=darkblue,
        filecolor=black,
        urlcolor=darkblue,
        anchorcolor=black,
        menucolor=black,
        breaklinks=true,
        pageanchor=true, %for jumping to a page
        plainpages=false,
        pdfpagelabels=true]{hyperref}
    \pdfcompresslevel=9
    \pdfoutput=1
    \DeclareGraphicsExtensions{.pdf,.png,.jpg,.jpeg}
\else
    \usepackage{graphicx}
\fi
\usepackage{rotating} % rotate figures
\usepackage{subcaption}
\usepackage{wrapfig}


\usepackage{fancyhdr}
\pagestyle{fancy}
%\addtolength{\headwidth}{\marginparsep} %these change header-rule width
%\addtolength{\headwidth}{\marginparwidth}
\lhead{}
\chead{\small\scshape On the Origin of Elements in the Milky Way} 
\rhead{} 
\lfoot{} 
\cfoot{\thepage} 
\rfoot{} 
\renewcommand{\headrulewidth}{.3pt} 
\renewcommand{\footrulewidth}{.3pt}

% Redefine author as topic
\newcommand{\topic}{\author}

%
%Redefining sections as problems
%
\makeatletter
\newenvironment{problem}{\@startsection
    {section}
    {1}
    {-.2em}
    {-3.5ex plus -1ex minus -.2ex}
    {2.3ex plus .2ex}
    {
        \pagebreak[3] % forces pagebreak when space is small; use \eject for better results
        \noindent\sffamily\bfseries Problem
    }
}
{
    %\vspace{1ex}\begin{center} \rule{0.3\linewidth}{.3pt}\end{center}}
    \begin{center}\large\bfseries\ldots\ldots\ldots\end{center}
}
\makeatother

% set enumerate to use letters
\renewcommand{\theenumi}{\alph{enumi}}

% newcommands
%============
% my short cuts
\providecommand{\e}[1]{\ensuremath{\times 10^{#1}}}
\providecommand{\ex}[1]{\ensuremath{^{#1}}}
\providecommand{\dex}[1]{\ensuremath{\delta^{#1}}}
\newcommand{\nean}{$^{22}$Ne($\alpha$,n)$^{25}$Mg}

% textnormal
\newcommand{\tn}{\textnormal}
% textregistered
\newcommand{\tr}{$^\tn{\textregistered}$}


%-------------------DOCUMENT---------------------------

\begin{document}


\title{Homework \#5}
\topic{Mass Spectrometry, Stardust}
\date{Assigned: March 22, 2021 \qquad Due: March 29, 2021}

\maketitle
\thispagestyle{fancy}

\noindent\emph{Percentages for each problem of the total grade (100\%) as given. Sub-problems, if present, split the problem's percentage equally. Please show your work!}

\begin{problem}{Time-of-Flight Mass Analyzer (20\%)}
In a time-of-flight (TOF) mass analyzer, ions are separated by mass over charge based on the flight time they each take to travel a given distance $d$. After passing an electrical potential ($U$), all ions are given the same energy $E_\mathrm{el} = qU$. Depending on their mass, they will fly through the mass analyzer at different velocities. Show that the flight time is proportional to $\sqrt{m/q}$ and determine the proporitonality constant. What quantities go into the proportianlity constant?
\end{problem}

\begin{problem}{Delta-Values (20\%)}
The solar abundances of \ex{46}Ti and \ex{48}Ti are 204 and 1820 (normalized such that silicon is equal to $10^6$).
\begin{enumerate}
    \item Calculate the $\delta^{46}\mathrm{Ti}_{48}$ for a sample that shows a titanium isotope ratio of $^{46}\mathrm{Ti}/{^{48}}\mathrm{Ti} = 0.2$.
    \item What would it mean if a study reports $\delta^{46}\mathrm{Ti}_{48} = -1000 \permil$?
\end{enumerate}
\end{problem}

\begin{problem}{Solar System Contamination (20\%)}
The \textit{s}-process model by Lugaro et al. (2018), see Figure 7.10 in the lecture notes, predicts for a $M=2.5\,M_\odot$, $Z=2Z_\odot$ star the following $\delta$-values for the zirconium isotopic composition in the last thermal pulse in which most of the mass-loss happens:
\begin{itemize}
    \item $\delta^{92}\mathrm{Zr}_{94} = -207\permil$
    \item $\delta^{96}\mathrm{Zr}_{94} = -937\permil$
\end{itemize}
Assume that the measured presolar grian composition is a mixture of this \textit{s}-process component and some solar contamination. This contamination would have solar composition. In the Solar System, the relative abundances of \ex{92}Zr, \ex{94}Zr, and \ex{96}Zr are 17.146\%, 17.38\%, and 2.799\%, respectively. Create a $\delta$-plot analogous to Figure 7.10 where you show (1) the \textit{s}-process composition, (2) the solar composition, and (3) a mixing line in between the two. \emph{Hint:} Mixing is not linear in $\delta$-units, thus you must calculate mixing with isotope ratios.
\end{problem}

\begin{problem}{Number of Atoms per Presolar Grain (20\%)}
Assume that you have a spherical SiC grain with $1\,\mu$m radius. The density of SiC is $3.21$\,g\,cm$^{-3}$ and the molar mass about 40\,g\,mol$^{-1}$. 
\begin{enumerate}
    \item With Avogadro's constant $N_A=6.022\times10^{23}$\,mol$^{-1}$, calculate the number of SiC molecules in the grain.
    \item Assume that iron (molar mass: 56\,g\,mol$^{-1}$) has a concentration in the grain of 10\,ppm by weight. Calculate the number of iron atoms in the grain. \textit{Hint:} Be careful to not mix up weight fractions and number fractions!
\end{enumerate}
\end{problem}

\begin{problem}{Solar System SiC? (20\%)}
The constituents of the Solar System condensed from a hot molecular cloud. Explain why all SiC grains found in meteorites are bona fide stardust samples. Why did no SiC condense in the solar nebula?
\end{problem}

\end{document}
