\documentclass[letterpaper,12pt,twoside=false,DIV=11]{scrartcl}

%----------------------CONFIG---------------------------
%math packages
\usepackage{amsmath,amssymb,amsthm,units,unitsdef}

%bibliography style and citation style, bibstyles to use: plainnat, abbrvnat, unsrtnat, named, chicago
%otherwise numerical citationstyle will be used
%\usepackage[authoryear,round]{natbib}

\usepackage{longtable,tabularx,tabulary,multirow,lscape}
\usepackage[font={sl},format=plain,labelfont=bf]{caption}

% colors
\usepackage{color,colortbl}
\usepackage[dvipsnames]{xcolor}
\definecolor{darkblue}{HTML}{00354C}

\usepackage{booktabs}
%\usepackage{showkeys} % shows the labels above the references for

%easier development
\usepackage{ifpdf}

\ifpdf
    \usepackage[pdftex]{graphicx}
    \usepackage[]{pdfpages} %for including full pdf pages
    \usepackage[pdftex,
        bookmarks,
        bookmarksopen=true,
        bookmarksnumbered=true,
        pdfauthor={Reto Trappitsch},
        pdftitle={On the origin of elements in the Milky Way - Homework},
        colorlinks,
        linkcolor=darkblue,
        citecolor=darkblue,
        filecolor=black,
        urlcolor=darkblue,
        anchorcolor=black,
        menucolor=black,
        breaklinks=true,
        pageanchor=true, %for jumping to a page
        plainpages=false,
        pdfpagelabels=true]{hyperref}
    \pdfcompresslevel=9
    \pdfoutput=1
    \DeclareGraphicsExtensions{.pdf,.png,.jpg,.jpeg}
\else
    \usepackage{graphicx}
\fi
\usepackage{rotating} % rotate figures
\usepackage{subcaption}
\usepackage{wrapfig}


\usepackage{fancyhdr}
\pagestyle{fancy}
%\addtolength{\headwidth}{\marginparsep} %these change header-rule width
%\addtolength{\headwidth}{\marginparwidth}
\lhead{}
\chead{\small\scshape On the Origin of Elements in the Milky Way} 
\rhead{} 
\lfoot{} 
\cfoot{\thepage} 
\rfoot{} 
\renewcommand{\headrulewidth}{.3pt} 
\renewcommand{\footrulewidth}{.3pt}

% Redefine author as topic
\newcommand{\topic}{\author}

%
%Redefining sections as problems
%
\makeatletter
\newenvironment{problem}{\@startsection
    {section}
    {1}
    {-.2em}
    {-3.5ex plus -1ex minus -.2ex}
    {2.3ex plus .2ex}
    {
        \pagebreak[3] % forces pagebreak when space is small; use \eject for better results
        \noindent\sffamily\bfseries Problem
    }
}
{
    %\vspace{1ex}\begin{center} \rule{0.3\linewidth}{.3pt}\end{center}}
    \begin{center}\large\bfseries\ldots\ldots\ldots\end{center}
}
\makeatother

% set enumerate to use letters
\renewcommand{\theenumi}{\alph{enumi}}

% newcommands
%============
% my short cuts
\providecommand{\e}[1]{\ensuremath{\times 10^{#1}}}
\providecommand{\ex}[1]{\ensuremath{^{#1}}}
\providecommand{\dex}[1]{\ensuremath{\delta^{#1}}}
\newcommand{\nean}{$^{22}$Ne($\alpha$,n)$^{25}$Mg}

% textnormal
\newcommand{\tn}{\textnormal}
% textregistered
\newcommand{\tr}{$^\tn{\textregistered}$}


%-------------------DOCUMENT---------------------------

\begin{document}


\title{Homework \#2 -- Solution}
\topic{Big Bang Nucleosynthesis, Stellar Evolution}

\maketitle
\thispagestyle{fancy}


\begin{problem}{Primordial versus Stellar Helium (20\%)}

First let us convert the energy that is released when fusing 4 H to \ex{4}He to Joules. 1\,eV is equal to $1.602\times10^{-19}$\,J, thus the total $\Delta E = 4.534 \times 10^{-12}$\,J. The total amount of energy produced during the Sun's lifetime ($t_\odot$) assuming constant luminosity is simply
\begin{equation}
    E_\mathrm{tot} = L_\odot t_\odot.
\end{equation}
The number of \ex{4}He atoms can then be written as
\begin{equation}
    n_\mathrm{4He} = \frac{L_\odot t_\odot}{\Delta E}.
\end{equation}
This value, multiplied by the mass of \ex{4}He of $m_\mathrm{4He} = 6.646 \times 10^{-27}$\,kg. In order to get the mass fraction in terms of the mass of the Sun that has been produced during the whole lifetime of the Sun, we can write
\begin{equation}
    \frac{M_\mathrm{4He}}{M_\odot} = n_\mathrm{4He} m_\mathrm{4He} = \frac{L_\odot t_\odot m_\mathrm{4He}}{\Delta E M_\odot} \approx 4\%.
\end{equation}

This shows that only little helium is made in stars, thus, the majority of helium indeed must have formed in the Big Bang.

\end{problem}

\begin{problem}{Distances and redshifts (20\%)}

\begin{enumerate}
    \item Equation (2.6) of the lecture notes shows Hubbles law. We know the speed $\dot{R}$ at which the Tadpole galaxy is moving away, and Hubble's constant $H_0$ is given as $70\,$km\,s$^{-1}$\,Mpc$^{-1}$. We can thus calculate the distance $R$ as
    \begin{equation}
        R = \frac{\dot{R}}{H_0} = 134\,\mathrm{Mpc}.
    \end{equation}
    Since 1\,Mpc is equal to 3.26\,Mly, we can calculate the distance to the Tadpole galaxy as 437 million light years.

    \item From equation (2.6) and (2.8) in the lecture notes we can write
    \begin{equation}
        \frac{\lambda_0}{\lambda} = \frac{\lambda_r}{\lambda_s} = \sqrt{\frac{1+\beta}{1-\beta}} = 1+z.
    \end{equation}
    With the speed of light ($c=3\times10^{5}$\,km\,s$^{-1}$), the factor $\beta = \frac{v}{c}$ can be calculated for the Tadpole Galaxy. This results in
    \begin{align}
        \frac{\lambda_0}{\lambda} &= 1.0318\\
        \Rightarrow z &= 0.0318.
    \end{align}
\end{enumerate}


\end{problem}


\begin{problem}{Jeans Radius for the Solar System (20\%)} 
    Solving equation (3.8) for $R$ as the Jeans radius we get
    \begin{equation}
        R = \frac{G\mu M_\odot}{5kT}. \label{eqn:rjeans}
    \end{equation}
    To calculate the average mass fraction of the gas we know that hydrogen occurs as H$_2$ molecules. By mass fractions we have approximately 75\% hydrogen and 25\%He. The number ratio of hydrogen to helium in the Solar System is thus $n_\mathrm{H}/n_\mathrm{He} = 12$ and we can calculate the number ratio of H$_2$ to He as
    \begin{equation}
        \frac{n_\mathrm{H2}}{n_\mathrm{He}} = 6.
    \end{equation}
    Calculating the average mass per molecule $\mu$ then yields
    \begin{equation}
        \mu = \frac{6 m_\mathrm{H2} + m_\mathrm{He}}{7} \approx 2.3\,\mathrm{amu}.
    \end{equation}
    Plugging all values into equation~\eqref{eqn:rjeans} we get for the Jeans radius
    \begin{equation}
        R = 7.3\times10^{14}\,\mathrm{m} \approx 5\times10^{3}\,\mathrm{AU}.
    \end{equation}

\end{problem}

\begin{problem}{Kepler's Third Law of Planetary Motion (20\%)}

    The virial theorem states that kinetic and potential energy are related such that
    \begin{equation}
        E_\mathrm{pot} + 2E_\mathrm{kin} = 0.
    \end{equation}
    For a planet in motion we can write the two energies as
    \begin{align}
        E_\mathrm{pot} &= G\frac{Mm}{r^2}\\
        E_\mathrm{kin} &= \frac{1}{2}mv^2.
    \end{align}
    Here, $m$ is the mass of the planet, $M$ the mass of the central body (Sun in the case of the Solar System), and $r$ the orbital distance. Plugging these equations into the virial theorem yields
    \begin{align}
       -G\frac{Mm}{r^2} + mv^2 &= 0 \\
       -G\frac{M}{r^2} + v^2 = 0. \label{eqn:kepler_1}
    \end{align}
    The orbital velocity is related to radius via the body's orbital period $T$ such that
    \begin{equation}
        v = \frac{2\pi r}{T}.
    \end{equation}
    Plugging this into equation~\eqref{eqn:kepler_1} and solving for $T^2/r^3$ yields
    \begin{equation}
        \frac{T^2}{r^3} = \frac{4\pi^2}{GM}.
    \end{equation}
    The gravitational constant $G$ times the mass of the central body $M$ is constant, thus showing that the $T^2/r^3$ is indeed constant for all planets.

\end{problem}


\begin{problem}{The day(s) the Earth Stood Still (20\%)}

    The described scenario is equivalent to the free fall time discussed in the lecture notes in Section~3.1.2. We can simply solve equation (3.11) for $\tau_\mathrm{ff}$, which results in
    \begin{equation}
        \tau_\mathrm{ff} = \sqrt{\frac{T_P^{2}}{32}}.
    \end{equation}
    Knowing the period of Earth's regular orbit to be $T_P = 365.24\,$d, we can calculate the free fall time as
    \begin{equation}
        \tau_\mathrm{ff} = 64.6\,\mathrm{d}.
    \end{equation}

\end{problem}

\end{document}
