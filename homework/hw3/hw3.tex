\documentclass[letterpaper,12pt,twoside=false,DIV=11]{scrartcl}

%----------------------CONFIG---------------------------
%math packages
\usepackage{amsmath,amssymb,amsthm,units,unitsdef}

%bibliography style and citation style, bibstyles to use: plainnat, abbrvnat, unsrtnat, named, chicago
%otherwise numerical citationstyle will be used
%\usepackage[authoryear,round]{natbib}

\usepackage{longtable,tabularx,tabulary,multirow,lscape}
\usepackage[font={sl},format=plain,labelfont=bf]{caption}

% colors
\usepackage{color,colortbl}
\usepackage[dvipsnames]{xcolor}
\definecolor{darkblue}{HTML}{00354C}

\usepackage{booktabs}
%\usepackage{showkeys} % shows the labels above the references for

%easier development
\usepackage{ifpdf}

\ifpdf
    \usepackage[pdftex]{graphicx}
    \usepackage[]{pdfpages} %for including full pdf pages
    \usepackage[pdftex,
        bookmarks,
        bookmarksopen=true,
        bookmarksnumbered=true,
        pdfauthor={Reto Trappitsch},
        pdftitle={On the origin of elements in the Milky Way - Homework},
        colorlinks,
        linkcolor=darkblue,
        citecolor=darkblue,
        filecolor=black,
        urlcolor=darkblue,
        anchorcolor=black,
        menucolor=black,
        breaklinks=true,
        pageanchor=true, %for jumping to a page
        plainpages=false,
        pdfpagelabels=true]{hyperref}
    \pdfcompresslevel=9
    \pdfoutput=1
    \DeclareGraphicsExtensions{.pdf,.png,.jpg,.jpeg}
\else
    \usepackage{graphicx}
\fi
\usepackage{rotating} % rotate figures
\usepackage{subcaption}
\usepackage{wrapfig}


\usepackage{fancyhdr}
\pagestyle{fancy}
%\addtolength{\headwidth}{\marginparsep} %these change header-rule width
%\addtolength{\headwidth}{\marginparwidth}
\lhead{}
\chead{\small\scshape On the Origin of Elements in the Milky Way} 
\rhead{} 
\lfoot{} 
\cfoot{\thepage} 
\rfoot{} 
\renewcommand{\headrulewidth}{.3pt} 
\renewcommand{\footrulewidth}{.3pt}

% Redefine author as topic
\newcommand{\topic}{\author}

%
%Redefining sections as problems
%
\makeatletter
\newenvironment{problem}{\@startsection
    {section}
    {1}
    {-.2em}
    {-3.5ex plus -1ex minus -.2ex}
    {2.3ex plus .2ex}
    {
        \pagebreak[3] % forces pagebreak when space is small; use \eject for better results
        \noindent\sffamily\bfseries Problem
    }
}
{
    %\vspace{1ex}\begin{center} \rule{0.3\linewidth}{.3pt}\end{center}}
    \begin{center}\large\bfseries\ldots\ldots\ldots\end{center}
}
\makeatother

% set enumerate to use letters
\renewcommand{\theenumi}{\alph{enumi}}

% newcommands
%============
% my short cuts
\providecommand{\e}[1]{\ensuremath{\times 10^{#1}}}
\providecommand{\ex}[1]{\ensuremath{^{#1}}}
\providecommand{\dex}[1]{\ensuremath{\delta^{#1}}}
\newcommand{\nean}{$^{22}$Ne($\alpha$,n)$^{25}$Mg}

% textnormal
\newcommand{\tn}{\textnormal}
% textregistered
\newcommand{\tr}{$^\tn{\textregistered}$}


%-------------------DOCUMENT---------------------------

\begin{document}


\title{Homework \#3}
\topic{The Sun}
\date{Assigned: March 1, 2021 \qquad Due: March 8, 2021}

\maketitle
\thispagestyle{fancy}

\noindent\emph{Percentages for each problem of the total grade (100\%) as given. Sub-problems, if present, split the problem's percentage equally. Please show your work!}

\begin{problem}{The Salpeter process (20\%)}
\begin{enumerate}
    \item Create a figure comparing the energy generation rates of the pp-chain, the CNO cycle, and the triple-$\alpha$ reaction (similar to Figure 4.5 in the lecture notes). Assume a core density for the Sun of $\rho = 150$\,g\,cm$^{-3}$ and a triple-$\alpha$ screening factor of 1. Hint: If you are using python, the script to generate Figure 4.5 in the lecture notes is available on \href{https://github.com/galactic-forensics/lecture_origin_elements/blob/main/figures/sun/energy_h_burning.py}{GitHub}.
    \item Discuss why the triple-$\alpha$ process does not take place in the center of the Sun.
    \item 5\% bonus: What is the physical meaning of the triple-$\alpha$ screening factor? 
\end{enumerate}
\end{problem}

\begin{problem}{Why do we have carbon (20\%)}
The triple-$\alpha$ process is a three-body reaction, which is very rare. Research and discuss why it does in fact happen in stars.
\end{problem}

\begin{problem}{Fate of the Earth (20\%)}
How hot will the Earth get when the Sun dies? Multiple answers are possible here, please reason according to your answer.
\end{problem}

\begin{problem}{Helix nebula (20\%)}
The Helix nebula is a planetary nebula with an angular diameter of $16'$. It is located around 213\,pc from Earth.
\begin{enumerate}
    \item Sketch the distance measurement and calculate the diameter of the Helix nebula.
    \item Assuming a constant expansion rate of the nebula gases of 20\,km\,s$^{-1}$, calculate its age.
\end{enumerate}

\noindent\emph{(Problem adopted after problem 13.8 in Carroll \& Ostlie (2017), ``An Introduction to Modern Astrophysics'', 2nd edn. (Cambridge University Press)\footnote{doi: \href{http://doi.org/10.1017/9781108380980}{\texttt{10.1017/9781108380980}}})}
\end{problem}

\begin{problem}{Population II stars (20\%)}
We have frequently discussed ultra-low metallicity, population II stars. Some examples of these stars are halo stars or the members of Reticulum II. 
\begin{enumerate}
    \item Calculate the maximum mass of population II stars assuming they formed around 1\,Ga after the Big Bang.
    \item Explain why these stars are difficult to observe.
\end{enumerate}
\end{problem}

\end{document}
