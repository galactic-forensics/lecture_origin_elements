\documentclass[letterpaper,12pt,twoside=false,DIV=11]{scrartcl}

%----------------------CONFIG---------------------------
%math packages
\usepackage{amsmath,amssymb,amsthm,units,unitsdef}

%bibliography style and citation style, bibstyles to use: plainnat, abbrvnat, unsrtnat, named, chicago
%otherwise numerical citationstyle will be used
%\usepackage[authoryear,round]{natbib}

\usepackage{longtable,tabularx,tabulary,multirow,lscape}
\usepackage[font={sl},format=plain,labelfont=bf]{caption}

% colors
\usepackage{color,colortbl}
\usepackage[dvipsnames]{xcolor}
\definecolor{darkblue}{HTML}{00354C}

\usepackage{booktabs}
%\usepackage{showkeys} % shows the labels above the references for

%easier development
\usepackage{ifpdf}

\ifpdf
    \usepackage[pdftex]{graphicx}
    \usepackage[]{pdfpages} %for including full pdf pages
    \usepackage[pdftex,
        bookmarks,
        bookmarksopen=true,
        bookmarksnumbered=true,
        pdfauthor={Reto Trappitsch},
        pdftitle={On the origin of elements in the Milky Way - Homework},
        colorlinks,
        linkcolor=darkblue,
        citecolor=darkblue,
        filecolor=black,
        urlcolor=darkblue,
        anchorcolor=black,
        menucolor=black,
        breaklinks=true,
        pageanchor=true, %for jumping to a page
        plainpages=false,
        pdfpagelabels=true]{hyperref}
    \pdfcompresslevel=9
    \pdfoutput=1
    \DeclareGraphicsExtensions{.pdf,.png,.jpg,.jpeg}
\else
    \usepackage{graphicx}
\fi
\usepackage{rotating} % rotate figures
\usepackage{subcaption}
\usepackage{wrapfig}


\usepackage{fancyhdr}
\pagestyle{fancy}
%\addtolength{\headwidth}{\marginparsep} %these change header-rule width
%\addtolength{\headwidth}{\marginparwidth}
\lhead{}
\chead{\small\scshape On the Origin of Elements in the Milky Way} 
\rhead{} 
\lfoot{} 
\cfoot{\thepage} 
\rfoot{} 
\renewcommand{\headrulewidth}{.3pt} 
\renewcommand{\footrulewidth}{.3pt}

% Redefine author as topic
\newcommand{\topic}{\author}

%
%Redefining sections as problems
%
\makeatletter
\newenvironment{problem}{\@startsection
    {section}
    {1}
    {-.2em}
    {-3.5ex plus -1ex minus -.2ex}
    {2.3ex plus .2ex}
    {
        \pagebreak[3] % forces pagebreak when space is small; use \eject for better results
        \noindent\sffamily\bfseries Problem
    }
}
{
    %\vspace{1ex}\begin{center} \rule{0.3\linewidth}{.3pt}\end{center}}
    \begin{center}\large\bfseries\ldots\ldots\ldots\end{center}
}
\makeatother

% set enumerate to use letters
\renewcommand{\theenumi}{\alph{enumi}}

% newcommands
%============
% my short cuts
\providecommand{\e}[1]{\ensuremath{\times 10^{#1}}}
\providecommand{\ex}[1]{\ensuremath{^{#1}}}
\providecommand{\dex}[1]{\ensuremath{\delta^{#1}}}
\newcommand{\nean}{$^{22}$Ne($\alpha$,n)$^{25}$Mg}

% textnormal
\newcommand{\tn}{\textnormal}
% textregistered
\newcommand{\tr}{$^\tn{\textregistered}$}


%-------------------DOCUMENT---------------------------

\begin{document}


\title{Homework \#7}
\topic{\textit{p}-nuclei formation, \textit{r}-process}
\date{Assigned: April 19, 2021 \qquad Due: April 26, 2021}

\maketitle
\thispagestyle{fancy}

\noindent\emph{Percentages for each problem of the total grade (100\%) as given. Sub-problems, if present, split the problem's percentage equally. Please show your work!}

\begin{problem}{The Distance of Gamma-Ray Bursts (20\%)}
Assume that the measured energy fluence of a gamma-ray burst is $10^{-7}$\,J\,m$^{-2}$. Calculate the energy of the original event assuming that:
\begin{enumerate}
    \item The source is in the Oort cloud of comets within our Solar System at a distance of 50\,kAU.
    \item The source is extragalactic at a distance of 1\,Gpc.
\end{enumerate}
Compare these energies to each other and to other stellar events. 

\noindent\emph{(Problem adopted after example 4.1 in Carroll \& Ostlie (2017), ``An Introduction to Modern Astrophysics'', 2nd edn. (Cambridge University Press)\footnote{doi: \href{http://doi.org/10.1017/9781108380980}{\texttt{10.1017/9781108380980}}})}

\end{problem}

\begin{problem}{Uranium in the Early Solar System (20\%)}
Natural uranium consists of two isotopes that are long-lived enough to have survived the 4.567\,Ga since the formation of the Solar System. These are \ex{235}U with a half-life of $7.038\times10^8$\,a and \ex{238}U with a half-life of $4.468\times10^{9}$\,a. The current abundance of these uranium isotopes is 0.00724 and 0.992742, respectively.

\begin{enumerate}
    \item Calculate the uranium isotopic composition in the early Solar System.
    \item Assuming that the \textit{r}-process makes \ex{235}U and \ex{238}U in equal proportions, when did the \textit{r}-process take place that formed our uranium?
    \item Bonus 5\%: Would the \ex{235}U enrichment in the early Solar System be enough to drive a nuclear power plant without further enrichments? How about a nuclear weapon?
\end{enumerate}
\end{problem}

\begin{problem}{Primary vs. Secondary Processes (20\%)}
Explain the difference between a primary versus a secondary nucleosynthesis process. Discuss for the \textit{s}- and \textit{r}-process the category they fall into. Are the \textit{p}-nuclei formed in a primary or secondary process? Explain.
\end{problem}


\begin{problem}{\textit{s}-, \textit{r}-, and \textit{p}- nuclei (20\%)}
Give examples of 5 nuclei each that are \textit{s}-only, \textit{r}-only, and \textit{p}-only and state why these nuclei cannot be made by other processes. Hint: You can assume that \ex{100}Mo is almost exclusively an \textit{r}-only nucleus. 

A chart of the nuclides might help with this exercise. If you don't have a paper copy, you can find many free, online versions. A very detailed chart can be found on the website of the International Atomic Energy Agency (IAEA).\footnote{\url{https://www-nds.iaea.org/relnsd/vcharthtml/VChartHTML.html}}
\end{problem}


\begin{problem}{it (20\%)}
The neutron density in the \textit{r}-process is variable, but a typical value would be $\rho_n \sim 10^{22}\,\mathrm{cm}^{-3}$. One neutron has a mass of $m_n = 1.674\times10^{-27}$\,kg. Calculate the density of the neutrons and compare to water, which has a density of $1$\,g\,cm$^{-3}$. Why does the \textit{r}-process not take place in the ocean?
\end{problem}

\end{document}
