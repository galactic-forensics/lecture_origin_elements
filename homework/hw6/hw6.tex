\documentclass[letterpaper,12pt,twoside=false,DIV=11]{scrartcl}

%----------------------CONFIG---------------------------
%math packages
\usepackage{amsmath,amssymb,amsthm,units,unitsdef}

%bibliography style and citation style, bibstyles to use: plainnat, abbrvnat, unsrtnat, named, chicago
%otherwise numerical citationstyle will be used
%\usepackage[authoryear,round]{natbib}

\usepackage{longtable,tabularx,tabulary,multirow,lscape}
\usepackage[font={sl},format=plain,labelfont=bf]{caption}

% colors
\usepackage{color,colortbl}
\usepackage[dvipsnames]{xcolor}
\definecolor{darkblue}{HTML}{00354C}

\usepackage{booktabs}
%\usepackage{showkeys} % shows the labels above the references for

%easier development
\usepackage{ifpdf}

\ifpdf
    \usepackage[pdftex]{graphicx}
    \usepackage[]{pdfpages} %for including full pdf pages
    \usepackage[pdftex,
        bookmarks,
        bookmarksopen=true,
        bookmarksnumbered=true,
        pdfauthor={Reto Trappitsch},
        pdftitle={On the origin of elements in the Milky Way - Homework},
        colorlinks,
        linkcolor=darkblue,
        citecolor=darkblue,
        filecolor=black,
        urlcolor=darkblue,
        anchorcolor=black,
        menucolor=black,
        breaklinks=true,
        pageanchor=true, %for jumping to a page
        plainpages=false,
        pdfpagelabels=true]{hyperref}
    \pdfcompresslevel=9
    \pdfoutput=1
    \DeclareGraphicsExtensions{.pdf,.png,.jpg,.jpeg}
\else
    \usepackage{graphicx}
\fi
\usepackage{rotating} % rotate figures
\usepackage{subcaption}
\usepackage{wrapfig}


\usepackage{fancyhdr}
\pagestyle{fancy}
%\addtolength{\headwidth}{\marginparsep} %these change header-rule width
%\addtolength{\headwidth}{\marginparwidth}
\lhead{}
\chead{\small\scshape On the Origin of Elements in the Milky Way} 
\rhead{} 
\lfoot{} 
\cfoot{\thepage} 
\rfoot{} 
\renewcommand{\headrulewidth}{.3pt} 
\renewcommand{\footrulewidth}{.3pt}

% Redefine author as topic
\newcommand{\topic}{\author}

%
%Redefining sections as problems
%
\makeatletter
\newenvironment{problem}{\@startsection
    {section}
    {1}
    {-.2em}
    {-3.5ex plus -1ex minus -.2ex}
    {2.3ex plus .2ex}
    {
        \pagebreak[3] % forces pagebreak when space is small; use \eject for better results
        \noindent\sffamily\bfseries Problem
    }
}
{
    %\vspace{1ex}\begin{center} \rule{0.3\linewidth}{.3pt}\end{center}}
    \begin{center}\large\bfseries\ldots\ldots\ldots\end{center}
}
\makeatother

% set enumerate to use letters
\renewcommand{\theenumi}{\alph{enumi}}

% newcommands
%============
% my short cuts
\providecommand{\e}[1]{\ensuremath{\times 10^{#1}}}
\providecommand{\ex}[1]{\ensuremath{^{#1}}}
\providecommand{\dex}[1]{\ensuremath{\delta^{#1}}}
\newcommand{\nean}{$^{22}$Ne($\alpha$,n)$^{25}$Mg}

% textnormal
\newcommand{\tn}{\textnormal}
% textregistered
\newcommand{\tr}{$^\tn{\textregistered}$}


%-------------------DOCUMENT---------------------------

\begin{document}


\title{Homework \#6}
\topic{Classical Novae, Monte Carlo Error Propagation}
\date{Assigned: March 31, 2021 \qquad Due: April 14, 2021}

\maketitle
\thispagestyle{fancy}

\noindent\emph{Percentages for each problem of the total grade (100\%) as given. Sub-problems, if present, split the problem's percentage equally. Please show your work!}

\begin{problem}{Proper Pressure and Mass Ejections (20\%)}
\begin{enumerate}
    \item For a white dwarf with a fixed size of $\approx 0.01\,R_\odot$, plot the accreted mass ($M_\mathrm{accr}$) as a function of the white dwarf mass ($M_\mathrm{WD}$) in the range $1\,M_\odot \leq M_\mathrm{WD} \leq 1.4\,M_\odot$. Assume a proper pressure of $10^{19}$\,Pa. What is the trend and how do you interpret it?
    \item What is the total accreted mass for a $1.2\,M_\odot$ white dwarf that is necessary for the thermonuclear runaway to occur? What is the average density of this white dwarf compared to the Sun's average density?
\end{enumerate}
\end{problem}

\begin{problem}{Recurrent Timescale for Classical Nova (20\%)}
Using the accreted mass calculated above for the $1.2\,M_\odot$ white dwarf and mass accretion rate of $10^{-11}\,M_\odot\,\mathrm{a}^{-1} \leq \dot{M} \leq 10^{-7}\,M_\odot\,\mathrm{a}^{-1}$, calculate how often you would expect such a nova to occur. How is it possible that recurrent nova happen with time intervals of 10\,a to 100\,a?
\end{problem}

\begin{problem}{Classical Novae versus Type Ia Supernovae (20\%)}
We have seen that classical novae and type Ia supernovae have very similar origins. Both events take place in binary star systems. Discuss the difference between a classical nova and a SN-Ia, assuming the latter takes place in the framework of the single-degenerate scenario.
\end{problem}

\begin{problem}{Monte Carlo Error Propagation (40\%)}
In the \href{https://github.com/galactic-forensics/lecture_origin_elements}{GitHub repository} for the class you can find a Jupyter notebook in the \texttt{hw6} folder. The notebook is named \texttt{Introduction to MC in Python.ipynb}. \href{https://github.com/galactic-forensics/lecture_origin_elements/blob/main/homework/hw6/Introduction%20MC%20in%20Python.ipynb}{Here is also a direct link to the notebook}. This notebook gives an introduction to Monte Carlo error propagation. Follow through the notebook.

At the end of the notebook there are three exercises, the third one being a bonus exercise. Solve these exercises in Python. Feel free to submit your solution as a Jupyter notebook.

If you are having problems running the provided Jupyter notebook on your own computer, you can use the Astrohub\footnote{\url{https://astrohub.uvic.ca}} that you already used for homework \#4. You can upload the provided Jupyter notebook there via drag-and-drop into the file menu. Please let me know if you encounter any issues!
\end{problem}

\end{document}
