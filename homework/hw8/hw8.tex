\documentclass[letterpaper,12pt,twoside=false,DIV=11]{scrartcl}

%----------------------CONFIG---------------------------
%math packages
\usepackage{amsmath,amssymb,amsthm,units,unitsdef}
\usepackage{listings}

%bibliography style and citation style, bibstyles to use: plainnat, abbrvnat, unsrtnat, named, chicago
%otherwise numerical citationstyle will be used
%\usepackage[authoryear,round]{natbib}

\usepackage{longtable,tabularx,tabulary,multirow,lscape}
\usepackage[font={sl},format=plain,labelfont=bf]{caption}

% colors
\usepackage{color,colortbl}
\usepackage[dvipsnames]{xcolor}
\definecolor{darkblue}{HTML}{00354C}

\usepackage{booktabs}
%\usepackage{showkeys} % shows the labels above the references for

%easier development
\usepackage{ifpdf}

\ifpdf
    \usepackage[pdftex]{graphicx}
    \usepackage[]{pdfpages} %for including full pdf pages
    \usepackage[pdftex,
        bookmarks,
        bookmarksopen=true,
        bookmarksnumbered=true,
        pdfauthor={Reto Trappitsch},
        pdftitle={On the origin of elements in the Milky Way - Homework},
        colorlinks,
        linkcolor=darkblue,
        citecolor=darkblue,
        filecolor=black,
        urlcolor=darkblue,
        anchorcolor=black,
        menucolor=black,
        breaklinks=true,
        pageanchor=true, %for jumping to a page
        plainpages=false,
        pdfpagelabels=true]{hyperref}
    \pdfcompresslevel=9
    \pdfoutput=1
    \DeclareGraphicsExtensions{.pdf,.png,.jpg,.jpeg}
\else
    \usepackage{graphicx}
\fi
\usepackage{rotating} % rotate figures
\usepackage{subcaption}
\usepackage{wrapfig}


\usepackage{fancyhdr}
\pagestyle{fancy}
%\addtolength{\headwidth}{\marginparsep} %these change header-rule width
%\addtolength{\headwidth}{\marginparwidth}
\lhead{}
\chead{\small\scshape On the Origin of Elements in the Milky Way} 
\rhead{} 
\lfoot{} 
\cfoot{\thepage} 
\rfoot{} 
\renewcommand{\headrulewidth}{.3pt} 
\renewcommand{\footrulewidth}{.3pt}

% Redefine author as topic
\newcommand{\topic}{\author}

%
%Redefining sections as problems
%
\makeatletter
\newenvironment{problem}{\@startsection
    {section}
    {1}
    {-.2em}
    {-3.5ex plus -1ex minus -.2ex}
    {2.3ex plus .2ex}
    {
        \pagebreak[3] % forces pagebreak when space is small; use \eject for better results
        \noindent\sffamily\bfseries Problem
    }
}
{
    %\vspace{1ex}\begin{center} \rule{0.3\linewidth}{.3pt}\end{center}}
    \begin{center}\large\bfseries\ldots\ldots\ldots\end{center}
}
\makeatother

% set enumerate to use letters
\renewcommand{\theenumi}{\alph{enumi}}

% newcommands
%============
% my short cuts
\providecommand{\e}[1]{\ensuremath{\times 10^{#1}}}
\providecommand{\ex}[1]{\ensuremath{^{#1}}}
\providecommand{\dex}[1]{\ensuremath{\delta^{#1}}}
\newcommand{\nean}{$^{22}$Ne($\alpha$,n)$^{25}$Mg}

% textnormal
\newcommand{\tn}{\textnormal}
% textregistered
\newcommand{\tr}{$^\tn{\textregistered}$}


%-------------------DOCUMENT---------------------------

\begin{document}


\title{Homework \#8}
\topic{Galactic Chemical Evolution}
\date{Assigned: April 28, 2021 \qquad Due: May 12, 2021}

\maketitle
\thispagestyle{fancy}

\textbf{Please submit your answers to the questions below in writing. Please also submit a copy of the Jupyter notebook that contains your best case scenario, i.e., your optimized parameters (question f).}

This homework set focuses on exploring a simple GCE model. As in homework \#4, you will be using a Jupyter notebook that is available on AstroHub.\footnote{\url{https://astrohub.uvic.ca}, see homework \#4 for details.} 

The Jupyter notebook you will be using can be found in the following folder on AstroHub: \texttt{/wendi-examples/Galactic\_chemical\_evolution\_school/}. The file is named \texttt{HS\_GCE\_Step\_1\_Constrain\_MW\_model.ipynb}. You should execute this file on AstroHub in order to have all data available.

\textbf{Attention:} The notebook has a small error in it and won't produce plots in its current stage. Please put the following in a cell before the first plot: \texttt{\%matplotlib inline}. Then delete all lines that say \texttt{\%matplotlib nbagg}. Plotting should now work.

Please read through the whole notebook in order to understand the different assumptions, etc. You will be modifying the parameters for run 2, e.g., the following lines:

\lstset{language=Python}
\begin{lstlisting}
# Set of parameters for the second run
# !! Please modify whatever you want !!
sfe = 0.04
t_star = 1.0e8
in_mag = 1.0
\end{lstlisting}

Please work on the following questions:
\begin{enumerate}
    \item What do the parameters \texttt{sfe}, \texttt{t\_star}, and \texttt{in\_mag} represent?
    \item Set \texttt{t\_star} and \texttt{in\_mag} equal to the example and change \texttt{sfe}. How do the the SFR, the ISM gas mass, and the [Fe/H] vary over the age of the galaxy?
    \item Set \texttt{sfe} and \texttt{in\_mag} equal to the example and change \texttt{t\_star}. How do the the SFR, the ISM gas mass, and the [Fe/H] vary over the age of the galaxy?
    \item Set \texttt{sfe} and \texttt{t\_star} equal to the example and change \texttt{in\_mag}. How do the the SFR, the ISM gas mass, and the [Fe/H] vary over the age of the galaxy?
    \item Can you find a parameter set that brings the integrated stellar mass, the SFR, ISM gas mass, and the [Fe/H] into agreement with the Milky Way observations?
    \item Bonus 10\%: How does the inflow rate change with the parameters discussed above? (See Section on ``Extra Material''.)
\end{enumerate}

There are several more examples on AstroHub that show you more GCE details. Please feel free to look into them, be curious, play, and have fun!
\end{document}
