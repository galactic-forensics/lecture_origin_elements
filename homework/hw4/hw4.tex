\documentclass[letterpaper,12pt,twoside=false,DIV=11]{scrartcl}

%----------------------CONFIG---------------------------
%math packages
\usepackage{amsmath,amssymb,amsthm,units,unitsdef}

%bibliography style and citation style, bibstyles to use: plainnat, abbrvnat, unsrtnat, named, chicago
%otherwise numerical citationstyle will be used
%\usepackage[authoryear,round]{natbib}

\usepackage{longtable,tabularx,tabulary,multirow,lscape}
\usepackage[font={sl},format=plain,labelfont=bf]{caption}

% colors
\usepackage{color,colortbl}
\usepackage[dvipsnames]{xcolor}
\definecolor{darkblue}{HTML}{00354C}

\usepackage{booktabs}
%\usepackage{showkeys} % shows the labels above the references for

%easier development
\usepackage{ifpdf}

\ifpdf
    \usepackage[pdftex]{graphicx}
    \usepackage[]{pdfpages} %for including full pdf pages
    \usepackage[pdftex,
        bookmarks,
        bookmarksopen=true,
        bookmarksnumbered=true,
        pdfauthor={Reto Trappitsch},
        pdftitle={On the origin of elements in the Milky Way - Homework},
        colorlinks,
        linkcolor=darkblue,
        citecolor=darkblue,
        filecolor=black,
        urlcolor=darkblue,
        anchorcolor=black,
        menucolor=black,
        breaklinks=true,
        pageanchor=true, %for jumping to a page
        plainpages=false,
        pdfpagelabels=true]{hyperref}
    \pdfcompresslevel=9
    \pdfoutput=1
    \DeclareGraphicsExtensions{.pdf,.png,.jpg,.jpeg}
\else
    \usepackage{graphicx}
\fi
\usepackage{rotating} % rotate figures
\usepackage{subcaption}
\usepackage{wrapfig}


\usepackage{fancyhdr}
\pagestyle{fancy}
%\addtolength{\headwidth}{\marginparsep} %these change header-rule width
%\addtolength{\headwidth}{\marginparwidth}
\lhead{}
\chead{\small\scshape On the Origin of Elements in the Milky Way} 
\rhead{} 
\lfoot{} 
\cfoot{\thepage} 
\rfoot{} 
\renewcommand{\headrulewidth}{.3pt} 
\renewcommand{\footrulewidth}{.3pt}

% Redefine author as topic
\newcommand{\topic}{\author}

%
%Redefining sections as problems
%
\makeatletter
\newenvironment{problem}{\@startsection
    {section}
    {1}
    {-.2em}
    {-3.5ex plus -1ex minus -.2ex}
    {2.3ex plus .2ex}
    {
        \pagebreak[3] % forces pagebreak when space is small; use \eject for better results
        \noindent\sffamily\bfseries Problem
    }
}
{
    %\vspace{1ex}\begin{center} \rule{0.3\linewidth}{.3pt}\end{center}}
    \begin{center}\large\bfseries\ldots\ldots\ldots\end{center}
}
\makeatother

% set enumerate to use letters
\renewcommand{\theenumi}{\alph{enumi}}

% newcommands
%============
% my short cuts
\providecommand{\e}[1]{\ensuremath{\times 10^{#1}}}
\providecommand{\ex}[1]{\ensuremath{^{#1}}}
\providecommand{\dex}[1]{\ensuremath{\delta^{#1}}}
\newcommand{\nean}{$^{22}$Ne($\alpha$,n)$^{25}$Mg}

% textnormal
\newcommand{\tn}{\textnormal}
% textregistered
\newcommand{\tr}{$^\tn{\textregistered}$}


%-------------------DOCUMENT---------------------------

\begin{document}


\title{Homework \#4}
\topic{Massive stars, \emph{s}-process nucleosynthesis}
\date{Assigned: March 15, 2021 \qquad Due: March 22, 2021}

\maketitle
\thispagestyle{fancy}

\noindent\emph{Percentages for each problem of the total grade (100\%) as given. Sub-problems, if present, split the problem's percentage equally. Please show your work!}

\begin{problem}{Isotope Abundances in the Iron Peak (20\%)} \label{prob1}

With a binding energy of 8.7945\,MeV per nucleon, \ex{62}Ni is the most tightly bound nucleus. The isotopes \ex{58}Fe and \ex{56}Fe are the second and third most tightly bound ones with binding energies of 8.7922\,MeV per nucleon and 8.7903\,MeV per nucleon, respectively. Explain why \ex{56}Fe is so much more abundant in the Solar System compared to \ex{58}Fe and \ex{62}Ni.

\end{problem}

\begin{problem}{Supernova Explosion Energy (20\%)}

We have seen that supernovae release a total of around 100\,foe of energy when they explode. Starting with hydrogen and fusing it, how much \ex{56}Fe would you have to produce in order for the energy release to be similar? Express your result in ``useful'' units, i.e., in units that compare to the mass of an object you are familiar with (Earth, Sun, Jupiter, Moon, \dots). \emph{Hint:} You can use the binding energy for \ex{56}Fe as given in Problem~\ref{prob1}, but remember that that binding energy is given per nucleon.

\end{problem}

\begin{problem}{Local Approximation (20\%)}
Create a figure similar to Figure 6.2 in the lecture notes for the isotopes \ex{126}Xe, \ex{128}Xe, \ex{129}Xe, and \ex{130}Xe. You can use Maxwellian-averaged cross sections from KADoNiS\footnote{\url{https://kadonis.org}} and your preferred solar abundances.
\begin{enumerate}
    \item Explain why \ex{128}Xe and \ex{130}Xe adhere to the local approximation.
    \item Why does \ex{129}Xe not agree at all with the local approximation?
    \item 5\% Bonus: Explain why \ex{126}Xe does not agree with the local approximation, even though it is shiedled from the \textit{r}-process by \ex{126}Te.
\end{enumerate}
\end{problem}

\begin{problem}{Exploring a Star (40\%)}
The following exercise can be done using the Public Astrohub Instance of the NuGrid Collaboration. You will need a (free) GitHub\footnote{\url{https://github.com}} account in order to log in to the Astrohub. 
Navigate to \url{https://astrohub.uvic.ca/} and click on ``Public \& Outreach''. After signing in with your GitHub account, you should reach the spawner menu. You want to keep the option \texttt{Jupyter Lab}, choose the application \texttt{MESAHub: Run and analyse MESA/NuGrid simulations}, and click \texttt{spawn}. If the spawner menu does not appear but Jupyter lab starts directly, click \texttt{File}, \texttt{Hub Control Panel}, \texttt{Stop My Server} to reach the spawner menu. 

You should now see a Jupyter Lab environment. In the file menu, navigate to \texttt{wendi-examples/Stellar evolution and nucleosynthesis data/} and open the Notebook names \texttt{Star\_explore.ipynb}. This notebook shows you how to explore a $2\,M_\odot$, $Z_\odot$ star. The first part of the notebook explores the stellar evolution model, the second part the post-processed nucleosynthesis calculation. Go through the notebook and answer the following for questions about this star:
\begin{enumerate}
    \item The first figure, titled ``Figure 111'' shows a Kippenhahn diagram of the star. What is going on at each step in the black curve? 
    \item When does the star loose most of its mass? How much mass is lost in total?
    \item How big (in terms of Lagrangian mass coordinates) is the \ex{13}C-pocket in this stellar model? You can use ``Figure 124'' of the notebook to evaluate the pocket size. Note though that python gives the mass axis some strange factor on the right-most side. Explain your reasoning on how you determined the \ex{13}C-pocket size.
    \item Using ``Figure 123'', which is the overall most abundant isotope during \textit{s}-process nucleosynthesis and why? For masses $A>80$, which are the two most abundant isotopes? 
    \item Bonus 5\%: Following up the most abundant isotopes with masses $A>80$: Can you explain why this is expected?
\end{enumerate}

Feel free to play with the Jupyter Lab and explore other stars, etc. This exercise only gives you a glimpse, but you can modify all of these routines and create your own ones to browse the NuGrid data.
\end{problem}

\end{document}
